\documentclass[11 pt]{article}
\title{induction proof}
\usepackage{latexsym}
\usepackage{amssymb}
\usepackage{amsfonts}
\usepackage{amsmath}
\usepackage{amsthm}
\newtheorem{proposition}{Proposition}

\newcommand{\newpar}{\vspace{.15in}\noindent}

\begin{document}

\noindent Jake Irons, MTH 325-02, Proof 1

\newpar
\begin{proposition}
For all natural numbers, $n$, 3 divides $4^n - 1$.
\end{proposition}
\begin{proof}
We will prove the statement using mathematical induction. Let $P(n)$ be the predicate 3 divides $4^n-1$.

\newpar
When $n=0$, we obtain $4^0-1 = 0$. Since $0=3(0)$, by the definition of divides, 3 divides $4^0-1$. Therefore, $P(0)$ is true.

\newpar We now assume that $P(k)$ is true and show that $P(k+1)$ is true. That is, we assume our induction hypothesis, 3 divides $4^k-1$ for some integer k, and show that 3 divides $4^{(k+1)}-1$. By the definition of divides, there exists and integer $q$ such that $4^k-1=3q$. Therefore, we can determine $4^k=3q+1$. Using algebra we obtain the following:
\begin{align*}
4^{(k+1)}-1&=4^k\cdot4^1-1 \\
&= 4(3q+1)-1 \\
&= 12q+4-1 \\
&= 3(4q+1). \\
\end{align*}

\noindent
We will label $d=4q+1$. Because 4, 1, and $q$ are integers and the set of integers is closed under the operation of multiplication and addition, we know that $d$ is also an integer. Because $4^{(k+1)}-1=3d$, it can be determined by the definition of divides that 3 divides $4^{(k+1)}-1$.

\newpar
We have proven that if the predicate $P(k)$ is true, then so is $P(k+1)$. By the principal of mathematical induction, the predicate is true for every $n\in{\mathbb{N}}$, and the proof is complete.
\end{proof}



\end{document}